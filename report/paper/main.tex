\documentclass[a4paper]{article}

\usepackage[english]{babel}
\usepackage[utf8]{inputenc}
% \usepackage{amsmath}
% \usepackage{graphicx}
% \usepackage[colorinlistoftodos]{todonotes}
\usepackage{listings}
\usepackage{graphicx}
\usepackage{hyperref}
% \input{scaladoc.sty}
% % To activate Scala support for the listings package, include this file with:
% % To activate Scala support for the listings package, include this file with:
% % To activate Scala support for the listings package, include this file with:
% \input{scalamacros.tex}

% This file includes code from the Scala distribution (package scala-tool-support), hence it is released
% under a BSD-like license - original download page: http://www.scala-lang.org/downloads
% The license itself: http://www.scala-lang.org/node/146

\usepackage{listings}

% Merged from http://tihlde.org/~eivindw/latex-listings-for-scala/ and 
% http://lampsvn.epfl.ch/trac/scala/export/26099/scala-tool-support/trunk/src/latex/scaladoc.sty
% "define" Scala
%Keyword list taken from the scaladoc definition.
\lstdefinelanguage{scala}{
  morekeywords={%
          abstract,case,catch,class,def,do,else,extends,%
          false,final,finally,for,forSome,if,implicit,import,lazy,%
          match,new,null,object,override,package,private,protected,%
          return,sealed,super,this,throw,trait,true,try,type,%
          val,var,while,with,yield},
  otherkeywords={=>,<-,<\%,<:,>:,\#,@},
  sensitive=true,
  morecomment=[l]{//},
  morecomment=[n]{/*}{*/},
  morestring=[b]",
  morestring=[b]',
  morestring=[b]"""
}[keywords,comments,strings]

% activate the language and predefine settings
\lstset{language=Scala}

% This file includes code from the Scala distribution (package scala-tool-support), hence it is released
% under a BSD-like license - original download page: http://www.scala-lang.org/downloads
% The license itself: http://www.scala-lang.org/node/146

\usepackage{listings}

% Merged from http://tihlde.org/~eivindw/latex-listings-for-scala/ and 
% http://lampsvn.epfl.ch/trac/scala/export/26099/scala-tool-support/trunk/src/latex/scaladoc.sty
% "define" Scala
%Keyword list taken from the scaladoc definition.
\lstdefinelanguage{scala}{
  morekeywords={%
          abstract,case,catch,class,def,do,else,extends,%
          false,final,finally,for,forSome,if,implicit,import,lazy,%
          match,new,null,object,override,package,private,protected,%
          return,sealed,super,this,throw,trait,true,try,type,%
          val,var,while,with,yield},
  otherkeywords={=>,<-,<\%,<:,>:,\#,@},
  sensitive=true,
  morecomment=[l]{//},
  morecomment=[n]{/*}{*/},
  morestring=[b]",
  morestring=[b]',
  morestring=[b]"""
}[keywords,comments,strings]

% activate the language and predefine settings
\lstset{language=Scala}

% This file includes code from the Scala distribution (package scala-tool-support), hence it is released
% under a BSD-like license - original download page: http://www.scala-lang.org/downloads
% The license itself: http://www.scala-lang.org/node/146

\usepackage{listings}

% Merged from http://tihlde.org/~eivindw/latex-listings-for-scala/ and 
% http://lampsvn.epfl.ch/trac/scala/export/26099/scala-tool-support/trunk/src/latex/scaladoc.sty
% "define" Scala
%Keyword list taken from the scaladoc definition.
\lstdefinelanguage{scala}{
  morekeywords={%
          abstract,case,catch,class,def,do,else,extends,%
          false,final,finally,for,forSome,if,implicit,import,lazy,%
          match,new,null,object,override,package,private,protected,%
          return,sealed,super,this,throw,trait,true,try,type,%
          val,var,while,with,yield},
  otherkeywords={=>,<-,<\%,<:,>:,\#,@},
  sensitive=true,
  morecomment=[l]{//},
  morecomment=[n]{/*}{*/},
  morestring=[b]",
  morestring=[b]',
  morestring=[b]"""
}[keywords,comments,strings]

% activate the language and predefine settings
\lstset{language=Scala}
% "define" Scala
\lstdefinelanguage{scala}{
  morekeywords={abstract,case,catch,class,def,%
    do,else,extends,false,final,finally,%
    for,if,implicit,import,match,mixin,%
    new,null,object,override,package,%
    private,protected,requires,return,sealed,%
    super,this,throw,trait,true,try,%
    type,val,var,while,with,yield},
  otherkeywords={=>,<-,<\%,<:,>:,\#,@},
  sensitive=true,
  morecomment=[l]{//},
  morecomment=[n]{/*}{*/},
  morestring=[b]",
  morestring=[b]',
  morestring=[b]""",
  moredelim=**[is][\btHL]{`}{`},
}

% Default settings for code listings
% \lstset{language=scala,showstringspaces=false,columns=flexible, basicstyle=\footnotesize \ttfamily}
% \definecolor{ggray}{gray}{0.5}
% \renewcommand{\ttdefault}{pcr}
\lstset{
  % frame=tb,
  language=scala,
%   aboveskip=3mm,
%   belowskip=3mm,
%   showstringspaces=false,
%   columns=flexible,
  basicstyle={\footnotesize \ttfamily},
%   %basicstyle=\LSTfont,
% %% numbers:
% %  numbers=none,
%   numbers=left,
%   %xleftmargin=2em,
%   %framexleftmargin=1.5em,
% %%% end numbers
%   numberstyle=\tiny\color{gray},
  keywordstyle=\bfseries,
  % commentstyle=\em\color{ggray},
  stringstyle=\em,
% %  keywordstyle=\color{blue},
% %  commentstyle=\color{OliveGreen},
% %  stringstyle=\color{Purple},
%   frame=single,
%   breaklines=true,
%   breakatwhitespace=true
  tabsize=4
}


\title{Attributed parser combinators}

\author{Cédric Bastin}

\date{\today}

\newcommand{\pre}[1]{\begin{verbatim}#1\end{verbatim}}

\begin{document}

\maketitle

\centerline{\includegraphics[width=3cm]{epfl.png}}

\begin{abstract}
\section{Abstract}
Through the rise of so called \emph{Big Data} as well as the corresponding analytics, the requirements for fast and efficient data retrieval and manipulation are suggesting new advancement with parsers. Those conditions make parsing techniques regain importance in order to allow faster data access and treatment. However constructing parse trees can be expensive as not all of the parsed content might be of interest or if the data in its basic form is not useful and needs to be transformed anyway.
We will present a generic parsing framework in Scala which can be used to compose a formal grammar with different semantic functions who are applied directly during the parsing process. The \textbf{AGParsers} (attribute grammar parsers) use a polyvalent structure to allow to user to easily extend it and adapt it to his own use-cases.
We use ideas from attribute grammars to combine syntactic and semantic features during parsing thus creating a parsing framework which can carry partial result information through the parsing process which can be used directly during parsing. Thus the user can define exactly which structure he would like to be created by the parser without the need of several manipulating passes over the parse tree structure.
The described functionality is achieved by augmenting the existing scala parser combinator to be able to carry additional information during parsing which can be used and augmented depending on each parsing rule and the related semantic attribute function.

\end{abstract}

\section{Introduction \& motivation}
Parsing is still an interesting topic as programming languages evolve and some compiler developers might still be dreaming about a \textbf{one-pass compiler} or at least to apply several manipulating steps at once. Those steps require more knowledge about the tree structure in general which can be acquired by defining several recursive functions over the parse tree which pass around the additional information needed. Another method to achieve this is to use attribute grammar to augment the existing nodes in the tree with the needed information which can then be accessed several times from different locations. Attribute grammars allow to defined a set of attribute on a tree node, this information is then made available without changing the original tree structure.
Parser combinators have shown that it is possible to have an compostable and easy to use parsing framework which allows to quickly create parsers for an input of your choice without the need to write complicated parsers by hand nor to use parser generating tools. We wanted to achieve the same ease of use by providing a parsing framework with more functionality such that more manipulating steps can already be calculated during the parsing.

On the other hand nowadays parsing is used the most frequently in combination with information coming from the web such as html or xml where only part of the information is potentially useful which in the past could only be abstracted after constructing the full parse tree.

Tree structures are often used to structure data as they implicitly encode the relationship between different data points. Hence many tree manipulation techniques have been introduced to allow different applications such as traversers, transformers and collectors.

In each application it is possible that local knowledge is not enough for the needed functionality, for example a transformer might need global knowledge of the tree and a collector might need an environment to accumulate the different results. One concrete application would be context sensitive parsing which requires knowledge of previously seen data in order to continue parsing. Those additional parameters could in some cases be encoded into a tupled return type however is this solution not generic and requires to change the parser each time you want to extract a different information. An easier to use approach would allow the programmer to plug a different semantic function into the parser depending on his need.

Even though attributes grammar have been around for almost as long as functional programming they never really gained a lot of popularity. We wanted to revisit the related techniques to provide a more accessible framework to use them in combination with parsing.

Scala Combinator have been implemented and used in a variety of programming languages to facilitate the task of the parser writer to reduce it to a simple composition of smaller parsers

During the development of the project we considered various implementations of such a framework. One of which was based on TQL, developed by Eric Beguet, however did we find the use of Monoid a limiting factors as they don't help to see the full hierarchy of the tree and treats all the nodes in the same fashion.
We will also cover related work where some research has been done 

\section{Background}
\subsection{Attribute Grammars}
Attribute grammars were introduced by Donald Knuth in 1967. They formalize the a set of attributes over a formal grammar such that each production rule can have one or more attributes associated to it. Thus an attribute grammar is defined over the nodes of an abstract syntax tree such that they can be transformed into a corresponding value. Most those attributes are not simple local transformations but need extended knowledge of a larger subset of the parse tree.

There are 2 different kinds of attributes, the synthesized and inherited attributes. As the name suggest the inherited attributes are used to pass semantic information to the child branches of a node and synthesized attributes are used to pass semantic information up in the parse tree. As parsers are generally constructing a tree structure from a linear input they have to work left-to-right fashion and thus some of the needed information might potentially be unavailable at the time of the parsing of a node. Lazy evaluation in functional programming can be used to overcome these restrictions as long as no cyclic dependencies between then attribute dependencies occur.

Here are some examples of attributes grammars.
In this example the attribute \verb/value/ is used to evaluate the expressions on the fly during parsing:
\begin{verbatim}
Expr1 -> Expr2 + Term     [ Expr1.value = Expr2.value + Term.value ]
Expr -> Term              [ Expr.value = Term.value ]
\end{verbatim}
Another example would be the deapth in the tree of each node:
\begin{verbatim}
Node -> NodeL Value NodeR [ NodeL.depth = Node.depth + 1 ... ] 
Node -> Leaf              [ Leaf.depth = Node.depth + 1 ]
\end{verbatim}

Furthermore as the syntactic definition of a formal grammar might be more permissive than the actual language it is related to, an attribute grammar could be used to validate the parsed content and provide an additional wrapping mechanism for parse results. 

As the attributes are an abstract way of decorating the abstract syntax tree one can easily decouple the attribute rules from the grammar such that a different attribute grammar can be used over the same formal grammar providing different computations over a parse tree without the need to rewrite the grammar nor adapt the other attribute functions.

\subsection{Monads}
A monad in functional programming represents an encapsulated computation and its result such that it can be composed or pipelined with other monads. Monads are used to abstract over the side effects of a computation and avoid mutation. A Monad only consists of a type constructor as well as 2 operations namely \verb/unit/ and \verb/bind/ following the monadic laws of associativity and left/right identity. Those semantics allow monads to be easily composable and allow a large array of manipulations. The operations on a monad can access and modify or augment its content during a \textbf{map} call without giving up it's external type structure thus providing for example nested failure safety.
 
Monads can also be used for parsing as they allo easy composition of monadic parses. In general the use of parser combinators would lead to a nested structure of tuples of parse result which could either be a success or a failure. To avoid those nested structured and repetitive checks one can use a monad which would automatically cascade a parse failure without explicit handling or checking at each "level".\\
$type\, M\, a = String \rightarrow [(a, String)]$\\
The same technique has been used by the scala parser combinators which also use the monadic style for composition and result handling.

Monads can also be used to encapsulate state such that the programmer can include state information within each computed result, in non-functional programming this would represent information stored in variables which are not part of the parameters of a functions and thus need to be mutable.\\
$unit: T \rightarrow S \rightarrow T \times S $\\
Another related example is the reader monad (also called environment monad) which allows the pipelined monads to share an environment which they can read from and augment with new elements.\\
$unit: T \rightarrow E \rightarrow T $\\
for instance the result of reader together with the input still left to read. This has been described in the paper "Monads for functional programming"\cite{monads} by Philip Wadler. In general the state monad can carry any intermediate result of a partially applied evaluation whereas the reader contains individual collected result such as an environment of things encountered during a computation

For the creation of the augmented parser combinator framework some influence was taken from monad transformers which are type constructors which can combine 2 monads. This allows to compose monads in order to get a new monadic structure combining the features of the underlying ones. For instance the reader monad transformer:\\
$unit: A \rightarrow E \rightarrow M A $\\
It takes an environment and some other monad as input and apply the reader transformation to the content of the monad given as argument while maintaining the outer layer of the same type.  

\subsection{Parser combinators}
Compared to the clumsy parser generator tools which generate parser from a context free grammar, Scala took inspiration from Haskell and its parser combinators. Parser generators are a library DSL which can be used for compositional parsing using smaller building blocks.
Since Scala 2.11 they have been factored out of the scala language into a separate library which can be used in your projects by including them with sbt://
\begin{verbatim}libraryDependencies += "org.scala-lang.modules" %% "scala-parser-combinators" % "1.0.4"\end{verbatim}//
There are a set of combinator which can be used to create composed parser such as the sequential composition combinator \begin{verbatim}parserA ~ parserB\end{verbatim} which will return a ParseResult of both \verb/parserA/ and \verb/parserB/ concatenated (in a \begin{verbatim}~(a:A, b:B)\end{verbatim} object). It is easy to do pattern matching on those results as a syntactically sugared infix notation is available: a ~ b

\subsection{flatMap and context sensitive parsing}
One important parser combinator is the \verb/flatMap/ combinator, also called \verb/into/:
\begin{verbatim}def >>[U](fq: T => Parser[U]) = into(fq) = flatMap(fq)\end{verbatim}
This combinator helps to overcome some limitations of traditional parsers described earlier, namely it allows local context sensitivity during parsing. For instance does this method allow to extract the \begin{verbatim}message length\end{verbatim} information form a message header such that the body parser knows when to stop reading input. Another example would be a message dispatcher which would switch to use a different body parser implementation depending on the message type described in the header. Even though this is a very useful feature the scope is limited and thus it would be nice to have a way to propagate information through he full tree.

\section{Design \& Interface}
The implementation is based on top of the parser combinator library such that basic combinators can be reused and more advanced features such as token parsing do not need to be fully re-implemented. In our implementation we have augmented the parser combinators to include an environment which contains the attributes and can thus be read and augmented during the application of semantic functions.

\subsection{AGSig}
AGSignature describes the very general aspect of an abstract algebra, namely the abstract data type used to pass information between semantic functions. Also some general methods for specific Answer manipulations such as combination ca be defined here. As the framework was meant to be very generic we reduced the amount of those functions to the simple case of combination of 2 different Answers. Of course this abstract definition should be refined for each use case such that the 
\begin{lstlisting}
trait AGSig {
  type Answer
  def combine(a1:Answer, a2:Answer):Answer
}
\end{lstlisting}

When the framework is used we suggest to decouple semantic functions from the parsing grammar. This can be useful when a different set of attributes need to be used for different applications and we want to avoid copy-pasting the code of the grammar itself. 

\subsection{AGAlgebra}
\begin{lstlisting}
trait MyAlgebra extends AGSig {
  type Answer = List[String] //e.g. environment

  def start(tag:String, ps:List[Property], a:Answer):Answer = tag :: a
\end{lstlisting}

\subsection{AGGrammar}
\begin{lstlisting}
trait HtmlGrammar extends AGParsers with HtmlSig {
  lexical.delimiters ++= List("<", ">", "\\", """"""", "=")

  def Start: AGParser[(String, List[Property])] = {
    //(f: T => U, add:(Answer, U) => Answer)
    lift(keyword("<")) ~> lift(ident) ~ rep(PropertyP) <~ lift(keyword(">")) >>^^>> {
      case (tag ~ ps, a:Answer) => (start(tag, ps, a), (tag, ps))
    }
  }
\end{lstlisting}

As you can see the grammar immediately forwards the evaluation of the attributes to semantic functions of the algebra. 

\subsection{issues}
As the input for a parser is generally a linear structure we need to pipe the collected attributes collected in the *Answer* environment to the correct parser combinators. For instance when constructing an environment of all the parent nodes we want to make sure that the child nodes do not share the environment other than what comes from their common parent.

\begin{lstlisting}
def Leaf:AGParser = ValueParser
def Node:AGParser = Node ~ ValueParser ~ Node
\end{lstlisting}

We cannot simply propagate the environment from left to right as the collected values of the left hand side are not parents of the right hand side (we don't know actually). Thus we needed to a way to let the programmer express details about the structure being parsed and allow him to specify which and how the parser interrelate.

With the use of semantic functions in a parser one can easily see that the added value over syntactic information also requires for more in depth validation techniques as more structural information is available and this extended validation can take place.

\subsection{Implementation}
Due to the added environment for attribute results which needs to be passed around between parser combinators and possibly used or augmented, the standard combinators need to be adapted for this use. The changes mostly influence the *sequence* and *map* combinators.
We decided to use the double arrow notation *>>* to express whenever the environment is passed on between consecutive parsers and how the semantic functions applied in each step influence this Answer environment. 

\subsection{sequencing: ~ / \>>~ / ~>> / \>>~>>}
As explained above the inputed of a parser if often unstructured and linear and thus requires the programmer to encode the hierarchy and scope of a specific combinator manually. 
For instance the sequencing combinator can be used to combine 2 smaller parser, however each of those parser might potentially change the answer environment.
The given notation is not to be confused with the `~>` *drop-right* combinator!
For instance the "flow" of the Answer object of the operator `parserA >>~>> parserB` can be symbolized by `parserA >> parserB >> ...` which means that the potentially updated environment flows both between the 2 parsers and it is also passed on to whatever follow then. On the contrary `parserA ~>> parserB` means that only the answer from the second parser will be used in consecutive parsers
The full interaction can be expressed in a graph for and easy to understand illustration of how the different *answer piping* interrelate.
<img src="ans_piping.png" width="500px" style="display: block;margin-left: auto;margin-right: auto;"\>
Some confusion might occur due to the left associativity of parser combinators which makes some grouping implicit and tough to see which environment is actually piped where.
The programmer might need to use parentheses in order to keep the environment in the correct scope and make his attempt more clear.


\subsection{mapping: \begin{lstlisting}^^ / \>>^^ / ^^>> / \>>^^>>\end{lstlisting}}
Whenever a mapping function is applied on the result of a parser this functionality might influence the attribute environment which thus needs updating. On the other hand some semantic functions applied to a parse result might also need to access the attribute environment in order to apply a specific functionality. We used the same notation to express the different between mapping function which need to access the environment in order to apply the semantic function and the ones that augment the existing environment with new attribute values.
The function signatures look as follows:

\begin{lstlisting}
def ^^[U](f: T => U) = map(f)
def >>^^[U](f: (T, Answer) => U) = mapWithAns(f)
def ^^>>[U](f: T => U, add:(Answer, U) => Answer) = mapIntoAns(f, add)
def >>^^>>[U](f: (T, Answer) => (Answer, U)) = mapWithAnsIntoAns(f)
\end{lstlisting}

\subsection{ParseResult}
Obviously the attributes included the *Answer* need to be captured somewhere such that they can be passed around implicitly whenever they are not used. This is achieved by including the Answer inside the ResultType itself such that it can either be accessed or untouched depending on which attributes and evaluation it need to during the parsing.

\begin{lstlisting}
  case class AGSuccess[+T](result: T, next: Input, ans: Answer) extends AGParseResult[T]
  case class AGFailure(msg: String, next: Input) extends AGParseResult[Nothing]
\end{lstlisting}

This allows us to have monad-like structure and semantics such that the content of Answer can potentially change a Success to a Failure only depending on semantic features.

\section{Examples}
The use of the framework is best illustrated by concrete examples which explain the use of the \verb/Answer/ environment. We will show several examples to explain the use of AGParser as collector and transformer over parse trees.

\subsection{Typing}
Calculating the type of an expression is a typical case of a transformation which will return different information of a syntactic tree than what's given at the input.

Note that the given code only focuses on the details of the environment treatment, other important details were left out, please check out the code on github for the full implementation.

The abstract signature just contains the signatures of the possible semantic function and their abstract return type representing an attribute.
\begin{lstlisting}
trait StlcSig extends AGSig {
  def vari(s:String, a:Answer): Answer
  def absHead(ident:String, ty:Type): Answer
  def abs(tp:Answer, a:Answer): Answer
}
\end{lstlisting}

When looking at the parsers we just consider the interesting cases where the environment variables names and their corresponding types is either augmented or read. We can see that the \verb/abstraction/ parser had to be split into 2 parts such that the first parser is responsible of augmenting the environment before handing it over to the next parser.
We can see that we have to use the \verb/mapWithAns/ to type a variable using the existing environment and \verb/mapIntoAns/ to augment the existing environment to include a new variable-type mapping.

\begin{lstlisting}
trait StlcGrammar extends AGParsers with StlcSig {
  def AbsHead: AGParser[Answer] = {
    lift("\\") ~ lift(ident) ~ lift(":") ~ TypePars ~ lift(".") ^^>>[Answer] ({
      case "\\" ~ x ~ ":" ~ tp ~ "." => absHead(x, tp)
    }, combine)
  }

  def SimpleTerm: AGParser[Answer] = {
    ...
      | AbsHead >>~ Term ^^ { //pipe augmented answer with environment
      case head ~ term =>
        abs(head, term)
    } | lift(failure("illegal start of simple term"))
  }
}
\end{lstlisting}

In the concrete algebra we can see the corresponding semantic functions which are use to either augment the environment or extract information from it.

\begin{lstlisting}
trait SltcTypingAlgebra extends StlcSig {
  type Env = Map[String, Type]
  case class Answer(env: Env, tpe: Type)

  def vari(s:String, a:Answer): Answer = {
    a.env.get(s) match {
      case Some(t) => a.copy(tpe = t) //copy the type from the environment
      case None => a.copy(tpe = WrongType("could not be found"))
    }
  }

  def absHead(ident:String, ty:Type): Answer = {
    Answer(Map(ident -> ty), ty) //use combine operator to fuse two answers 
  }

  def abs(a1:Answer, a2:Answer): Answer = {
    combine(a1, a2)
  }
}
\end{lstlisting}

\subsection{RepMin}
Repmin is a famous example where, with a tree given as input we want to create a tree of the same shape where each value in the nodes corresponds to the global minimum. 
We use it as a prime example to show that the result of the semantic functions can be pipelined through the tree. We think that it is an interesting example which shows that attributes can be used anywhere in the tree and that the constructed can be substantially different from the original object without ever constructing an intermediate representation.
This example was implemented on a tree of integers. In the given example the only attribute we need to keep track of is the minimum value found so far which converges to the global minimum. As the global minimum is not immediately available we cannot reconstruct the corresponding nodes based on local information. Due to this constraint we build a higher order semantic functions to return itself a function which will construct the final tree only once the minimum value is found. This creates some kind of call-back tree similar to continuation passing style where the  returned value is a function which needs to be fed with the global minimum. Once the whole tree as been read we also have found the global and can thus evaluate the call stack to build the final tree.

\begin{lstlisting}
trait RepMinSig extends AGSig {
  type TreeF = Answer => Tree //evaluate only when minimum is found

  def node(a1:TreeF, i:Answer, a2:TreeF):TreeF
  def leaf: TreeF
}
\end{lstlisting}

The signature gives an abstract high level definition of the computation one might want to perform on each node.

\begin{lstlisting}
trait RepMinGrammar extends AGParsers with RepMinSig {
  def Root:AGParser[Tree] = {
    TreeP >>^^ {case (t, ans) => t(ans)} | //fully evaluate at the end!
      lift(failure("tree not parsable!"))
  }
  def TreeP:AGParser[TreeF] = {NodeP | LeafP}
  def NodeP:AGParser[TreeF] = {
    lift("(") >>~>> TreeP >>~>> ValP >>~>> TreeP >>~>> lift(")") >>^^>> {
      case (s1 ~ a1 ~ a ~ a2 ~ s2, ans) => (combine(ans, a), node(a1, a, a2))
    }
  }
  def LeafP:AGParser[TreeF] = {
    lift("x") ^^^ leaf
  }
  def ValP:AGParser[Answer] = {
    lift(numericLit) >>^^>> {
      case (s, ans) =>
        val res = combine(ans, toAns(s)) //"add to environment"
        (res, res) //return value of the parser is not important
    }
  }
\end{lstlisting}
As you can see the root parser is used to capture the final answer and use it to evaluate the constructed function which will yield as a result a tree of the same shape as the original one with each value replaced with the minimum.

\begin{lstlisting}
trait RepMinAlgebra extends RepMinSig {
  type Answer = Int

  def node(l:TreeF, a:Answer, r:TreeF): TreeF = {case i:Int => Node(l(i), i, r(i))}
  def leaf = {case _:Int => Leaf}

  def combine(a1:Answer, a2:Answer):Answer = if (a1 < a2) a1 else a2
}
\end{lstlisting}

As described earlier an important use case would be a parser for xml which would be able to extract specific content from a document and also validate that matching opening and closing tags exist. This can already be achieved using flatMap or even the simple sequencing parser however we tried to take a different approach namely to stack the the opening tags and de-stack them whenever we find the corresponding closing tags, this helps us to avoid growing a too large call stack and keep a simple environment instead.

\subsection{Html}
TODO

\section{Related work}
(Jonas: object algebra to attribute grammar)
\subsection{Kiama}
*Kiama is a Scala library for language processing which allows analysis and transformation on structured data using formal languages processing paradigms such as attribute grammars and tree rewriting.*
Through the use of Scala macros, Kiama augments existing tree structures with named or unnamed attribute function which can then be used to evaluate local or global properties.
\begin{verbatim}def attr[T,U] (f : T => U) : CachedAttribute[T,U] = macro AttributionCoreMacros.attrMacro[T,U,CachedAttribute[T,U]]\end{verbatim}
This makes the relation between the parents and child nodes in a tree explicit and thus allows the different attribute method to access each other. We wanted to avoid constructing an additional structure on top of the tree structure and actually completely dismiss the original parse tree if it is not needed for afterwards. With the AGParser it is possible to create parent-child and child-parents calls however do they need to be explicit.
Kiama is also able to treat more general graphs and thus to handle cyclic references

\subsection{TQL}
The tree query language presented by Eric Beguet offered an interesting insight on tree traversals and transformers even though they have been applied in the less general context of Scala meta trees. It can use different traversal techniques such as top-down or bottom-up or even break a specific traversal depending on the 

Monad paper (Reader, Env)

\subsection{Generalising Tree Traversals to DAGs}
In their paper, Bahr and Axelsson present generalized recursion schemes based on attribute grammars which can be applied to trees as well as DAGs. The main issue with attribute grammars on DAGs is that the different attributes might be recomputed for nodes which share the same branches, the paper presents a method to avoid this and to treat all shared nodes only once. A generalized fold method from previous work not work for context-dependent functions on trees with sharing. Moreover in the presented method is completely agnostic on the underlying structure such that it works either on trees or DAGs.
The also provide a way to combine multiple semantic function in order to use several attributes at once.

\subsection{DynaProg}
Some inspiration has been taken from the work done by Thierry Coppey who created a dynamic programming framework which was able to create the most efficient parse tree for a series of matrix multiplications.
Most importantly we took inspiration of the decoupling between the syntactic and demantic parts, i.e. the separation of the framework into an abstract signature, a parsing grammar and some concrete algebra combining using the abstract definition such that it can be composed with any concrete algebra.

\section{Future work}
With the current framework all attributes need to be declared explicitly in the \verb/AttrEnv/ type (which might be a case class or heterogeneous tuple) as they are not directly associated with the nodes themselves but rather present some global information of the parsing process which can be used and augmented by consecutive parsers.
As the parsing process generally works from left to right the only way to use attributes from a right hand side parser (e.g. value of a tree node) is by using the the lazy evaluation workaround used by the \verb/RepMin/ example.
Using several attribute algebras at once is still tedious and requires manual composition as the attribute type and the correspondong semantic functions needs to be adapted by hand to carry and manipulate the values of both computations.

In the current implementation the full parse tree is replace by the attributes, if this was to be avoided then we would need to include both the original return value as well as the manipulated one in the ParseResult. This would however also mean that the programmer needs to handle those cases explicitly which would change the interface from the traditional scala parser combinator framework.

\section{Acknowledgement}
I would like to thank my supervisors Sandro and Mano for the very interesting and insightful discussions. Through weekly meetings we were able to explore the different options as well a limitations caused by either the programming language or the choice of implementation such that the development of the project also helped to explore the theory behind parsing and attribute grammars.

\begin{thebibliography}{9}
\bibitem{tql}
  Traversal Query Language For Scala.Meta
  Eric Beguet, EPFL,
  2014
\bibitem{fastparsers}
  Accelerating parser combinators with macros
  Eric Beguet, EPFL,
  \url{https://github.com/begeric/FastParsers}
  2014
\bibitem{kiama}
  Kiama,
  \emph{A Scala library for language processing}
  \url{https://code.google.com/p/kiama/}
\bibitem{scalapc}
  Scala Parser Combinators,
  \url{https://github.com/scala/scala-parser-combinators}
\bibitem{attrurl}
  Why attribute grammars matter,
  The Monad Reader,
  \url{https://wiki.haskell.org/The_Monad.Reader/Issue4/Why_Attribute_Grammars_Matter}
  2005.
\bibitem{haskell}
  Generalising Tree Traversals to DAGs,
  Patrick Bahr, Emil Axelsson,
  2015.
\bibitem{monadic}
  Monadic Parser Combinator,
  Graham Hutton, Erik Meijer,
  1996.
\bibitem{monads}
  Monads for functional programming,
  Philip Wadler, University of Glasgow
  1995.
\bibitem{dynaprog}
  DynaProg for Scala,
  \emph{A Scala DSL for Dynamic Programming on CPU and GPU},
  Thierry Coppey, Master Thesis,
  2013.
\bibitem{knuth}
  Semantics of context-free languages,
  Don Knuth,
  1967.
\end{thebibliography}


% \section{Introduction}

% Your introduction goes here! Some examples of commonly used commands and features are listed below, to help you get started. If you have a question, please use the help menu (``?'') on the top bar to search for help or ask us a question.

% \section{Some \LaTeX{} Examples}
% \label{sec:examples}

% \subsection{How to Leave Comments}

% Comments can be added to the margins of the document using the \todo{Here's a comment in the margin!} todo command, as shown in the example on the right. You can also add inline comments:

% \todo[inline, color=green!40]{This is an inline comment.}

% \subsection{How to Include Figures}

% First you have to upload the image file (JPEG, PNG or PDF) from your computer to writeLaTeX using the upload link the project menu. Then use the includegraphics command to include it in your document. Use the figure environment and the caption command to add a number and a caption to your figure. See the code for Figure \ref{fig:frog} in this section for an example.

% \begin{figure}
% \centering
% \includegraphics[width=0.3\textwidth]{frog.jpg}
% \caption{\label{fig:frog}This frog was uploaded to writeLaTeX via the project menu.}
% \end{figure}

% \subsection{How to Make Tables}

% Use the table and tabular commands for basic tables --- see Table~\ref{tab:widgets}, for example.

% \begin{table}
% \centering
% \begin{tabular}{l|r}
% Item & Quantity \\\hline
% Widgets & 42 \\
% Gadgets & 13
% \end{tabular}
% \caption{\label{tab:widgets}An example table.}
% \end{table}

% \subsection{How to Write Mathematics}

% \LaTeX{} is great at typesetting mathematics. Let $X_1, X_2, \ldots, X_n$ be a sequence of independent and identically distributed random variables with $\text{E}[X_i] = \mu$ and $\text{Var}[X_i] = \sigma^2 < \infty$, and let
% $$S_n = \frac{X_1 + X_2 + \cdots + X_n}{n}
%       = \frac{1}{n}\sum_{i}^{n} X_i$$
% denote their mean. Then as $n$ approaches infinity, the random variables $\sqrt{n}(S_n - \mu)$ converge in distribution to a normal $\mathcal{N}(0, \sigma^2)$.

% \subsection{How to Make Sections and Subsections}

% Use section and subsection commands to organize your document. \LaTeX{} handles all the formatting and numbering automatically. Use ref and label commands for cross-references.

% \subsection{How to Make Lists}

% You can make lists with automatic numbering \dots

% \begin{enumerate}
% \item Like this,
% \item and like this.
% \end{enumerate}
% \dots or bullet points \dots
% \begin{itemize}
% \item Like this,
% \item and like this.
% \end{itemize}
% \dots or with words and descriptions \dots
% \begin{description}
% \item[Word] Definition
% \item[Concept] Explanation
% \item[Idea] Text
% \end{description}

% We hope you find write\LaTeX\ useful, and please let us know if you have any feedback using the help menu above.

\end{document}
